\documentclass[12pt,a4paper]{article}
\usepackage[utf8]{inputenc}
\usepackage[spanish]{babel}
\usepackage{geometry}
\usepackage{graphicx}
\usepackage{tikz}
\usepackage{listings}
\usepackage{xcolor}
\usepackage{float}
\usepackage{amsmath}
\usepackage{booktabs}
\usepackage{fancyhdr}
\usepackage{textcomp}

\usetikzlibrary{shapes.geometric, arrows.meta, positioning}

\geometry{margin=2.5cm}
\setlength{\headheight}{15pt}
\pagestyle{fancy}
\fancyhf{}
\rhead{Escenario 3 - Sistemas Computacionales}
\lhead{Priorización Invertida}
\rfoot{\thepage}

% Configuración de código
\lstdefinestyle{terminal}{
    basicstyle=\ttfamily\small,
    backgroundcolor=\color{black!5},
    frame=single,
    breaklines=true,
    columns=fullflexible,
    keepspaces=true,
    showstringspaces=false
}

\tikzstyle{startstop} = [ellipse, minimum width=2.5cm, minimum height=1cm, text centered, draw=black!80, fill=red!30, line width=0.8pt, font=\small\bfseries]
\tikzstyle{process} = [rectangle, rounded corners, minimum width=3.5cm, minimum height=1cm, text centered, text width=3.2cm, draw=black!80, fill=blue!25, line width=0.8pt, font=\small]
\tikzstyle{decision} = [diamond, minimum width=3cm, minimum height=1cm, text centered, text width=2.2cm, draw=black!80, fill=green!25, line width=0.8pt, aspect=2.5, font=\small]
\tikzstyle{arrow} = [thick,->,>=stealth,line width=1pt]
\tikzstyle{arrowlabel} = [font=\footnotesize,fill=white,inner sep=2pt]

\begin{document}

\begin{titlepage}
    \centering
    \vspace*{2cm}
    {\Huge\bfseries Lab5: Escenario 3\par}
    \vspace{1cm}
    \vspace{2cm}
    \vspace{2cm}
    \begin{tabular}{rl}
        \textbf{Profesora:} & Luz A. Adanaqué \\[0.3cm]
        \textbf{Alumno:} & Marco Soto 
    \end{tabular}
    \vfill
    {\large \today\par}
\end{titlepage}

\tableofcontents
\newpage

\section{Descripción General}

El Escenario 3 implementa un \textit{scheduler} satelital con \textbf{prioridad invertida} (P2 $>$ P1 $>$ P3), favoreciendo el proceso de enfriamiento antes de consultar el sensor o transmitir datos. Esta variante evalúa la rapidez de aplicación de técnicas de enfriamiento ante lecturas críticas y su impacto en pérdidas por conmutaciones abruptas.

\subsection{Características Principales}
\begin{itemize}
    \item \textbf{Orden de prioridad:} P2 (Enfriamiento) $>$ P1 (Sensor) $>$ P3 (UART)
    \item \textbf{Ciclo orbital:} 100 minutos (42 min luz / 58 min oscuridad)
    \item \textbf{Reacción inmediata:} Salto a P2 cuando T $\geq$ 100$^\circ$C
    \item \textbf{Registro de contexto:} PC guardado en cada cambio
\end{itemize}

\section{Diagrama de Flujo del Escenario}

El siguiente diagrama muestra la lógica de ejecución del scheduler con prioridades invertidas:

\begin{figure}[H]
\centering
\tikzset{
    startstop/.style={ellipse, minimum width=2cm, minimum height=0.7cm, text centered, draw=black!70, fill=red!25, line width=0.7pt, font=\tiny\bfseries},
    process/.style={rectangle, rounded corners=2pt, minimum width=2.5cm, minimum height=0.7cm, text centered, align=center, draw=black!70, fill=blue!20, line width=0.7pt, font=\tiny},
    decision/.style={diamond, minimum width=2.2cm, minimum height=0.9cm, text centered, align=center, draw=black!70, fill=green!20, line width=0.7pt, font=\tiny, aspect=2.2},
    alert/.style={rectangle, rounded corners=2pt, minimum width=2.5cm, minimum height=0.7cm, text centered, align=center, draw=red!70, fill=orange!25, line width=0.7pt, font=\tiny}
}
\begin{tikzpicture}[node distance=1.2cm, every node/.style={font=\tiny}, scale=0.85, transform shape]

% Nodos
\node[startstop] (A) {Inicio del sistema};
\node[process, below=of A] (B) {$t=0$ min};
\node[decision, below=of B] (C) {$t$ menor a\\100 min?};
\node[startstop, right=3.5cm of C] (Z) {Resumen final\\y metricas};

\node[process, below=of C] (D) {Scheduler (prioridad)\\P2$>$P1$>$P3};
\node[decision, below=of D] (F) {Proceso\\fue P1?};

\node[process, right=2.5cm of F] (K) {Siguiente segun\\round-robin};

\node[decision, below=of F] (G) {Temperatura\\$\geq$ 100C?};
\node[alert, below=of G] (H) {Salto ABRUPTO\\a P2 (anomalia)};
\node[process, below=of H] (I) {Registrar perdida\\de datos};
\node[process, below=of I] (J) {Incrementar\\interrupciones};

\node[process, below=of K] (M) {Cambio contexto\\normal};
\node[process, below=3.5cm of J] (L) {Context switch y\\actualizar PC};
\node[process, below=of L] (N) {Mostrar estado\\del sistema};
\node[process, below=of N] (O) {Incrementar tiempo\\en 5 min};

% Flechas
\path[-latex, line width=0.7pt]
    (A) edge (B)
    (B) edge (C)
    (C) edge node[above, font=\tiny, fill=white, inner sep=1pt] {No} (Z)
    (C) edge node[right, font=\tiny, fill=white, inner sep=1pt] {Si} (D)
    (D) edge (F)
    (F) edge node[above, font=\tiny, fill=white, inner sep=1pt] {No} (K)
    (F) edge node[right, font=\tiny, fill=white, inner sep=1pt] {Si} (G)
    (G) edge node[above, font=\tiny, fill=white, pos=0.3, inner sep=1pt] {No} (K)
    (G) edge node[right, font=\tiny, fill=white, inner sep=1pt] {Si} (H)
    (H) edge (I)
    (I) edge (J)
    (J) edge (L)
    (K) edge (M)
    (M) edge (L)
    (L) edge (N)
    (N) edge (O);

% Loop de retorno
\draw[-latex, line width=0.7pt] (O) -- ++(-4.5,0) |- (C);

\end{tikzpicture}
\caption{Flujo de ejecución del Escenario 3 con prioridades invertidas}
\label{fig:flowchart}
\end{figure}

\newpage
\section{Pruebas a Nivel de Terminal}

\subsection{Compilación y Ejecución}

El programa se compila y ejecuta en el simulador Spike con el siguiente procedimiento:

\begin{lstlisting}[style=terminal]
$ cd scenario3
$ ./compile.sh
$ spike --isa=rv64imac \
    /opt/homebrew/opt/riscv-pk/riscv64-unknown-elf/bin/pk \
    programa ../data/dataset_case2.txt
\end{lstlisting}

\subsection{Salida de Ejecución}

A continuación se muestra la ejecución completa del escenario con el dataset \texttt{case1}, el cual presenta un caso interesante con activación y desactivación del sistema de enfriamiento:

\begin{lstlisting}[style=terminal, basicstyle=\ttfamily\scriptsize]
=== ESCENARIO 3: Priorizacion invertida (P2 > P1 > P3) ===
Dataset cargado: ../data/dataset_case1.txt (64 muestras)

     t=0 min | Zona=Luminosa
[P1] t=  0 min | Temp=85 C | Zona=Luminosa | pc=1
        Cambio de contexto P1 -> P3
[OS] UART=0B pend | Cooling=OFF

     t=5 min | Zona=Luminosa
[P3] UART Transmission:
 - Temp: 85 C | Cooling: 0 | Zone: 1 (1=luz,0=oscuridad)
[P3] TX 16B (pendiente= 6B) | pc=1
        Cambio de contexto P3 -> P2
[OS] UART=6B pend | Cooling=OFF

     t=15 min | Zona=Luminosa
[P1] t= 15 min | Temp=92 C | Zona=Luminosa | pc=2
        Cambio de contexto P1 -> P3
[OS] UART=6B pend | Cooling=OFF

     t=25 min | Zona=Luminosa
[P2]     ACTIVADO (T>90 C)
        Cambio de contexto P2 -> P1
[OS] UART=0B pend | Cooling=ON

     t=30 min | Zona=Luminosa
[P1] t= 30 min | Temp=58 C | Zona=Luminosa | pc=3
        Cambio de contexto P1 -> P3
[OS] UART=0B pend | Cooling=ON

     t=40 min | Zona=Luminosa
[P2]     DESACTIVADO (T<60 C)
        Cambio de contexto P2 -> P1
[OS] UART=5B pend | Cooling=OFF

     t=45 min | Zona=Oscura
[P1] t= 45 min | Temp=93 C | Zona=Oscura | pc=4
        Cambio de contexto P1 -> P3
[OS] UART=5B pend | Cooling=OFF

     t=55 min | Zona=Oscura
[P2]     ACTIVADO (T>90 C)
        Cambio de contexto P2 -> P1
[OS] UART=0B pend | Cooling=ON

     t=60 min | Zona=Oscura
[P1] t= 60 min | Temp=97 C | Zona=Oscura | pc=5
        Cambio de contexto P1 -> P3
[OS] UART=0B pend | Cooling=ON

     t=70 min | Zona=Oscura
[P1] t= 70 min | Temp=104 C | Zona=Oscura | pc=6
        Cambio de contexto P1 -> P3
[OS] UART=5B pend | Cooling=ON

     t=95 min | Zona=Oscura
[P2]     DESACTIVADO (T<60 C)
        Cambio de contexto P2 -> P1
[OS] UART=0B pend | Cooling=OFF
\end{lstlisting}

\newpage
\subsection{Resumen de Ejecución}

Al finalizar la simulación, el sistema genera un resumen con las estadísticas de ejecución:

\begin{lstlisting}[style=terminal]
===== RESUMEN =====
Context switches: 20 | Abruptos: 0
Perdidas (B): P1=0, P3=0, P2=0
Interrupciones detectadas: 1

===== METRICAS ESCENARIO 3 =====
Texe total: 50.776 ms
Interrupciones por anomalias (T>=100 C): 1
Tiempo scheduler (busy loop): 50.006 ms
Proceso | Tiempo total (us) | Promedio (us) | Speedup (vs mas lento)
  P1    |           57.000  |         8.143 |   2.66 x
  P3    |          130.000  |        21.667 |   1.00 x
  P2    |           47.000  |         6.714 |   3.23 x

CPU Occupation: 98.94 %
Mem. Occupation: 0.45 KB (dataset + buffers + PCB)
\end{lstlisting}

\section{Explicación de Resultados y Métricas}

\subsection{Análisis de Cambios de Contexto}

La ejecución reportó \textbf{20 cambios de contexto}, todos normales (0 abruptos). Esto se debe a que:

\begin{itemize}
    \item El scheduler sigue estrictamente el orden de prioridad P2 $\to$ P1 $\to$ P3
    \item Los cambios son consecutivos según el \textit{round-robin} definido
    \item Solo se detectó 1 lectura con T $\geq$ 100°C (101°C en t=75 min), y el siguiente proceso en prioridad ya era P2
\end{itemize}

\subsection{Interrupciones por Anomalías}

Se detectó \textbf{1 interrupción} causada por temperatura mayor o igual a 100°C:

\begin{itemize}
    \item T = 101°C en t=75 min (zona oscura)
\end{itemize}

\textbf{Eventos de enfriamiento detectados:}
\begin{itemize}
    \item \textbf{Activación 1:} T=92°C en t=25 min (zona luminosa)
    \item \textbf{Desactivación 1:} T=58°C en t=40 min (zona luminosa)
    \item \textbf{Activación 2:} T=93°C en t=55 min (zona oscura)
    \item \textbf{Desactivación 2:} T=59°C en t=95 min (zona oscura)
\end{itemize}

\subsection{Métricas de Rendimiento}

\begin{table}[H]
\centering
\begin{tabular}{@{}lrrr@{}}
\toprule
\textbf{Proceso} & \textbf{Tiempo total ($\mu$s)} & \textbf{Promedio ($\mu$s)} & \textbf{Speedup} \\
\midrule
P1 (Sensor)      & 57.000 & 8.143 & 2.66x \\
P2 (Enfriamiento) & 47.000 & 6.714 & 3.23x \\
P3 (UART)        & 130.000  & 21.667 & 1.00x \\
\bottomrule
\end{tabular}
\caption{Tiempos de ejecución por proceso}
\label{tab:performance}
\end{table}

\textbf{Observaciones clave:}
\begin{itemize}
    \item \textbf{P3 se ejecutó correctamente}: Realizó 6 transmisiones UART durante la simulación
    \item \textbf{P3 es el proceso más lento}: Promedio de 21.667 $\mu$s por ejecución
    \item \textbf{P2 es el más rápido}: Speedup de 3.23x respecto a P3
    \item \textbf{Ocupación de CPU}: 98.94\% indica alta eficiencia del scheduler
    \item \textbf{Memoria}: Huella muy reducida de 0.45 KB
    \item \textbf{Tiempo total}: 50.776 ms de ejecución real para simular 100 minutos orbitales
\end{itemize}

\subsection{Pérdidas de Información}

No se registraron pérdidas de datos en ningún proceso:
\begin{itemize}
    \item P1: 0 bytes perdidos
    \item P2: 0 bytes perdidos
    \item P3: 0 bytes perdidos
\end{itemize}

Esto se debe a que todos los cambios de contexto fueron normales (consecutivos según prioridad) y no hubo saltos no consecutivos que causaran pérdidas.

\subsection{Comparación con Escenarios Previos}

\begin{table}[H]
\centering
\begin{tabular}{@{}lccc@{}}
\toprule
\textbf{Métrica} & \textbf{Esc. 1} & \textbf{Esc. 2} & \textbf{Esc. 3} \\
\midrule
Prioridad & P1$>$P2$>$P3 & P1$>$P2$>$P3 & P2$>$P1$>$P3 \\
Cambios abruptos & Variable & Variable & 0 \\
Interrupciones & Variable & Variable & 10 \\
Pérdidas totales (B) & Variable & Variable & 0 \\
CPU Occupation & -- & -- & 98.93\% \\
\bottomrule
\end{tabular}
\caption{Comparativa entre escenarios (valores del Esc. 3 con dataset case2)}
\label{tab:comparison}
\end{table}

\section{Conclusiones}

\begin{enumerate}
    \item La \textbf{priorización invertida} (P2 $>$ P1 $>$ P3) permite una respuesta inmediata a condiciones térmicas críticas al ejecutar el proceso de enfriamiento con mayor frecuencia.
    
    \item El diseño del scheduler logró \textbf{cero pérdidas de datos} al mantener cambios de contexto consecutivos, demostrando la eficiencia del algoritmo round-robin con prioridades.
    
    \item Se detectaron adecuadamente las \textbf{10 anomalías térmicas} (T $\geq$ 100°C), activando el sistema de enfriamiento según lo esperado.
    
    \item La alta \textbf{ocupación de CPU} (98.93\%) y baja huella de memoria (0.45 KB) demuestran un uso eficiente de recursos.
    
    \item El proceso P3 (UART) no se ejecutó en esta configuración, lo que podría indicar la necesidad de ajustar las condiciones de activación para escenarios específicos.
\end{enumerate}

\end{document}
